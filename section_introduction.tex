
\section*{Introduction}
A poster works just like a regular \LaTeX{} document.  You can create a section
hierarchy using:
\begin{itemize}
  \setlength{\itemindent}{1em}
  \item \verb+\section*{Introduction}+
  \item \verb+\subsection*{A subsection}+
  \item \verb+\subsubsection*{A sub sub section}+
  \item \verb+\paragraph*{A bolded text label for a paragraph}+
\end{itemize}

\noindent To customize various elements such as the:
\begin{description}
  \setlength{\itemindent}{1em}
  \item[Poster size], edit the \verb+papercustom.cfg+ file;
  \item[Title, authors, etc.], edit the top of the \verb+paper.tex+ file;
  \item[Title format], edit the \verb+title.tex+ file.
\end{description}

\noindent For more information on \LaTeX, view the the WikiBook \citet{wikibook:latex} as a reference.
If you are having trouble with something in \LaTeX, ask a question on \url{http://tex.stackexchange.com/}.

For this poster, a slight modification of the \verb+sciposter+ \citep{website:sciposter} package is used.
Refer to the \verb+sciposter+ manual \citep{manual:sciposter} for more detailed information on the package and the various options it offers.
